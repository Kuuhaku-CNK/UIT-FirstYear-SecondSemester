\documentclass[12pt]{article}
\usepackage[vietnamese]{babel}
\usepackage{graphicx}
\usepackage{listings}
\title{Vật lí bán dẫn và ứng dụng}
\author{Châu Nguyên Khang - MTIC2024}
\begin{document}
\setlength{\parindent}{10pt}
\maketitle
\newpage
\section{Chương 1: Cấu trúc tinh thể}
\subsection{Mạng tinh thể}
-Vật chất tồn tại 3 trạng thái cơ bản: Rắn, Lỏng, Khí
\begin{flushleft}
    \textbf{a) Về mặt cấu trúc:}
    \newline
    \textbf{+ Rắn (tinh thể + vô định hình):} Cấu trúc tinh thể có mật độ cấu trúc rất cao
    \newline
    \textbf{+ Khí:} cấu trúc hoàn toàn mất trật tự
    \newline
    \textbf{+ Lỏng:} Gần với cấu trúc tinh thể của rắn
\end{flushleft}
\begin{flushleft}
    \textbf{b) Cấu trúc tinh thể:}
    \smallskip
    \begin{center}
        Cấu trúc tinh thể  = mạng tinh thể  + cơ sở
    \end{center}
    \textbf{+ Tinh thể lý tưởng: } sự sắp xếp đều đặn trong một không gian đơn vị có cấu trúc giống nhau
    \newline
    \textbf{+ Đơn vị cấu trúc (cơ sở): } là một/nhiều nguyên tử , nhóm nguyên tử , phân tử
\end{flushleft}
\begin{flushleft}
    \textbf{c) Biểu diễn mạng tinh thể: }
    Tính tuần hoàn của mạng tinh thể
    
\end{flushleft}
\end{document}