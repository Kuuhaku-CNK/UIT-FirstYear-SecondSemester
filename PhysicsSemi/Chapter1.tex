\documentclass[12pt]{article}
\usepackage[vietnamese]{babel}
\usepackage{graphicx}
\usepackage{listings}
\title{Vật lí bán dẫn và ứng dụng}
\author{Châu Nguyên Khang - MTIC2024}
\begin{document}
\setlength{\parindent}{10pt}
\maketitle
\newpage
\section{Chương 1: Cấu trúc tinh thể}
\subsection{Mạng tinh thể}
-Vật chất tồn tại 3 trạng thái cơ bản: Rắn, Lỏng, Khí
    \newline
    \textbf{a) Về mặt cấu trúc:}
    \newline
    \begingroup
    \indent\textbf{+ Rắn (tinh thể + vô định hình):} Cấu trúc tinh thể có mật độ cấu trúc rấn
    \newline
    \indent\textbf{+ Khí:} cấu trúc hoàn toàn mất trật tự
    \newline
    \indent\textbf{+ Lỏng:} Gần với cấu trúc tinh thể của rắn
    \endgroup
    \newline
    \textbf{b) Cấu trúc tinh thể:}
    \begingroup
        \begin{center}
            Cấu trúc tinh thể = mạng tinh thể + cơ sở
        \end{center}
    \endgroup
    \begingroup
    \indent\textbf{+ Tinh thể lý tưởng: } sự sắp xếp đều đặn trong một không gian đơn vị có cấu trúc giống nhau
    \newline
    \indent\textbf{+ Đơn vị cấu trúc (cơ sở): } là một/nhiều nguyên tử , nhóm nguyên tử , phân tử
    \endgroup
    \newline
    \textbf{c) Biểu diễn mạng tinh thể: }
    \newline
    \begingroup
    \indent{+ Mạng tinh thể có tính tuần hoàn. Mọi nút của vector đều được suy ra từ nút gốc bằng phép tịnh tiến}
    \newline
    \indent{+ Công thức tịnh tiến:}
    \begin{equation}
        T = n_1a_1 + n_2a_2 + n_3a_3
    \end{equation}
    \indent Trong đó: 
    \newline
    \indent + \(a_1, a_2, a_3\) là 3 vector tịnh tiến không đồng phẳng \\
    \indent + T là vector tịnh tiến bảo toàn mạng tinh thể \\
    \indent + \(n_1, n_2, n_3\) là số nguyên hoặc phân số
    \newline
    \indent Nếu:

    \indent + \(n_1, n_2, n_3\) là \textbf{số nguyên} -> \(a_1, a_2, a_3\) là vector \textbf{nguyên tố}
    
    \indent + \(n_1, n_2, n_3\) là \textbf{phân số} -> \(a_1, a_2, a_3\) là vector \textbf{đơn vị}
    \endgroup
\end{document}